% Options for packages loaded elsewhere
\PassOptionsToPackage{unicode}{hyperref}
\PassOptionsToPackage{hyphens}{url}
\PassOptionsToPackage{dvipsnames,svgnames*,x11names*}{xcolor}
%
\documentclass[
  11pt,
  canadian,
  a4paper,
  open=right,
  twoside=true,
  cleardoublepage=empty,
  clearpage=empty]{scrbook}
\usepackage{lmodern}
\usepackage{amssymb,amsmath}
\usepackage{ifxetex,ifluatex}
\ifnum 0\ifxetex 1\fi\ifluatex 1\fi=0 % if pdftex
  \usepackage[T1]{fontenc}
  \usepackage[utf8]{inputenc}
  \usepackage{textcomp} % provide euro and other symbols
\else % if luatex or xetex
  \usepackage{unicode-math}
  \defaultfontfeatures{Scale=MatchLowercase}
  \defaultfontfeatures[\rmfamily]{Ligatures=TeX,Scale=1}
\fi
% Use upquote if available, for straight quotes in verbatim environments
\IfFileExists{upquote.sty}{\usepackage{upquote}}{}
\IfFileExists{microtype.sty}{% use microtype if available
  \usepackage[]{microtype}
  \UseMicrotypeSet[protrusion]{basicmath} % disable protrusion for tt fonts
}{}
\makeatletter
\@ifundefined{KOMAClassName}{% if non-KOMA class
  \IfFileExists{parskip.sty}{%
    \usepackage{parskip}
  }{% else
    \setlength{\parindent}{0pt}
    \setlength{\parskip}{6pt plus 2pt minus 1pt}}
}{% if KOMA class
  \KOMAoptions{parskip=half}}
\makeatother
\usepackage{xcolor}
\IfFileExists{xurl.sty}{\usepackage{xurl}}{} % add URL line breaks if available
\IfFileExists{bookmark.sty}{\usepackage{bookmark}}{\usepackage{hyperref}}
\hypersetup{
  pdftitle={Ultrafast energy flow in strongly-correlated materials},
  pdfauthor={Laurent P. René de Cotret},
  pdflang={en-CA},
  pdfkeywords={Markdown, Example},
  colorlinks=true,
  linkcolor=Maroon,
  filecolor=Maroon,
  citecolor=Blue,
  urlcolor=Blue,
  pdfcreator={LaTeX via pandoc}}
\urlstyle{same} % disable monospaced font for URLs
\usepackage[top=30mm,left=25mm,bottom=30mm,width=150mm,bindingoffset=6mm]{geometry}
\usepackage{graphicx}
\makeatletter
\def\maxwidth{\ifdim\Gin@nat@width>\linewidth\linewidth\else\Gin@nat@width\fi}
\def\maxheight{\ifdim\Gin@nat@height>\textheight\textheight\else\Gin@nat@height\fi}
\makeatother
% Scale images if necessary, so that they will not overflow the page
% margins by default, and it is still possible to overwrite the defaults
% using explicit options in \includegraphics[width, height, ...]{}
\setkeys{Gin}{width=\maxwidth,height=\maxheight,keepaspectratio}
% Set default figure placement to htbp
\makeatletter
\def\fps@figure{htbp}
\makeatother
\setlength{\emergencystretch}{3em} % prevent overfull lines
\providecommand{\tightlist}{%
  \setlength{\itemsep}{0pt}\setlength{\parskip}{0pt}}
\setcounter{secnumdepth}{5}

\usepackage{siunitx}
\usepackage[numbers]{natbib}
\usepackage{chapterbib}

% This is only to remove the title of bibliographies
% See here:
%   https://tex.stackexchange.com/questions/22645/hiding-the-title-of-the-bibliography
\renewcommand{\bibsection}{}

%------------------------------------- Custom Title Page ---------------------------
\renewcommand{\maketitle}{
\thispagestyle{empty}

\parindent=0pt

\begin{center}
    \LARGE
    \textbf{\textsc{ Ultrafast energy flow in strongly-correlated materials }}
\end{center}

\begin{center}
    \Large
    Laurent P. René de Cotret
\end{center}

\vspace{20mm}
\hrule
\vspace{5mm}

\begin{center}
    \Large
    \textsc{
        McGill University\\
        Faculty of Science\\
        %\vspace{\baselineskip}\noindent
        Department of Physics
    }
\end{center}

\vspace{5mm}
\hrule
\vspace{10mm}


\vfill

\vspace{\baselineskip}\noindent
\begin{center}
    \large
    April 15, 2021
\end{center}
}
%------------------------------------- Custom Title Page ---------------------------


% %------------------------------------- Custom Title (Back) Page --------------------
% \newcommand\customtitlebackpage{
% \thispagestyle{empty}
% \hfill
% \vfill
% \textbf{Laurent P. René de Cotret} (260446509) \\
% \textit{Ultrafast energy flow in strongly-correlated materials} \\
% April 15, 2021 \\[12pt]
% Erstprüfer/in:  \\
% Zweitprüfer/in: 
% }
% %------------------------------------- Custom Title (Back) Page --------------------




%------------------------------------- Patch Eisvogel's frontmatter ----------------
%% Eisvogel template uses `\frontmatter` directly *after* `\maketitle`, thus there is
%% a `\doubleclearpage` after the titlepage, which prevents our title-back-page to be
%% positioned on the backside of the titlepage ...
%% (Pandocs default template uses `\frontmatter` *before* `\maketitle`)
% %------------------------------------- Patch Eisvogel's frontmatter ----------------



%------------------------------------- Workaround for CleanStyle -------------------
%% cleanthesis.sty *will* check the bibfile, even if `configurebiblatex=false` ...
%% So we need to set it appropriately using our metadata variable "cleanthesisbibfile"
\PassOptionsToPackage{
    figuresep=colon,
    configurelistings=true,
    configurebiblatex=false,
    bibfile=references
}{cleanthesis}
%------------------------------------- Workaround for CleanStyle -------------------
%------------------------------------- Include Clean Thesis Style ------------------
%% Settings for cleanthesis.sty are defined in titlepage.tex since we need to set up
%% the bibfile using a metadata variable
%% We cannot put the `\usepackage{cleanthesis}` also in titlepage.tex since Pandoc
%% tries to read the local cleanthesis.sty.
\usepackage{cleanthesis}
%------------------------------------- Inc
\ifxetex
  % Load polyglossia as late as possible: uses bidi with RTL langages (e.g. Hebrew, Arabic)
  \usepackage{polyglossia}
  \setmainlanguage[variant=canadian]{english}
\else
  \usepackage[shorthands=off,main=canadian]{babel}
\fi
\usepackage[]{natbib}
\bibliographystyle{plainnat}
\newlength{\cslhangindent}
\setlength{\cslhangindent}{1.5em}
\newenvironment{cslreferences}%
  {}%
  {\par}

\title{Ultrafast energy flow in strongly-correlated materials}
\author{Laurent P. René de Cotret}
\date{April 15, 2021}

\begin{document}
\frontmatter
\maketitle


% %------------------------------------- Custom Title (Back) Page --------------------
% % \clearpage
% \customtitlebackpage
% % %------------------------------------- Custom Title (Back) Page --------------------



%------------------------------------- Custom Abstract Page ------------------------
\cleardoublepage
\begin{minipage}{\linewidth}

\chapter*{Abstract}
%\addcontentsline{toc}{chapter}{Abstract}
This is an english abstract.


\newpage
\chapter*{Résumé}
%\addcontentsline{toc}{chapter}{Résumé}
Ceci est le résumé.

\end{minipage}
\cleardoublepage
%------------------------------------- Custom Abstract Page ------------------------



%------------------------------------- Acknowledgements Page -----------------------
\chapter*{Aknowledgements}
%\addcontentsline{toc}{chapter}{Aknowledgements}
Acknowledgements go here.
%------------------------------------- Acknowledgements Page -----------------------


%------------------------------------- Clear page before TOC -----------------------
\cleardoublepage
%------------------------------------- Clear page before TOC -----------------------

%------------------------------------- Use Eisvogel Header/Footer ------------------
%------------------------------------- Use Eisvogel Header/Footer ------------------

{
\hypersetup{linkcolor=}
\setcounter{tocdepth}{2}
\tableofcontents
}
\mainmatter
\listoffigures

\listoftables

\hypertarget{sec:preface}{%
\chapter*{Preface}\label{sec:preface}}
\addcontentsline{toc}{chapter}{Preface}

The author declares that the work presented in this thesis constitutes
original scholarship and distinct contributions to the field of condensed matter physics.

The contribution of other members of the Siwick research group to this thesis is described.

The data exploration software package \texttt{iris} and its supporting libraries \texttt{crystals}, \texttt{npstreams}, and \texttt{scikit-ued}, have been conceptualized and realized by the author. The ultrafast electron diffractometer with which experiments were performed by the author was initially designed and built by Robert Chatelain and Vance Morisson, with significant advances in performance brought by Martin R. Otto. The data acquisition and reduction software that powers the instrument, \texttt{faraday}, was designed by the author. The ultrafast electron scattering experiments on graphite flakes were performed by Robert Chatelain, the density-functional perturbation theory calculations were conducted by Jan-Hendrick Pöhls; the author was responsible for the graphite data analysis and conclusions (Chapter \ref{sec:graphite}). The ultrafast electron scattering experiments on tin selenide were performed and analyzed by the author, while the time-resolved Terahertz spectroscopy measurements were led by Benjamin Dringoli and David G. Cooke, aided by the author (Chapter \ref{sec:snse}).

TODO: descriptions of papers for each chapter

\hypertarget{sec:introduction}{%
\chapter{Introduction}\label{sec:introduction}}

The fields of condensed matter physics and material science tackle a variety of intertwined questions, from fundamental physics to concrete applications. Given that condensed matter systems are complex, many experimental dimensions are used to distinguish between phenomena. Electron microscopy comes to mind as an example of a technique which can selectively observe physical systems in real-space (via imaging) or in momentum (via diffraction).

In highly-ordered systems (i.e.~crystals), where electronic correlation effects are most likely to matter, the time scales associated with fundamental actions are determined by the energy scale around room temperature. Phonons populated at room temperature (\SI{300}{\kelvin} = \SI{25}{\milli\electronvolt}) might have a period shorter than \SI{1}{\pico\second} (\(10^{-12}\) \si{\second}).

\hypertarget{sec:laser_sources}{%
\section{Ultrafast laser systems}\label{sec:laser_sources}}

Digital measurement devices are generally limited to nanosecond resolution, and are therefore too slow to observe dynamics in the femtosecond range. Therefore, at the heart of most ultrafast experiment lies an ultrafast laser system. An ultrafast laser system consists of a pulsed source of highly-coherent radiation.

Two things are required from the laser system: that its pulses be short, and that its pulses have high peak power. The length of laser pulses produced by such systems typically determine the time-resolution of experiments. The high peak power allows for the use of non-linear optical techniques, such as frequency-doubling.

Ultrafast laser systems like those used herein are composed of two optical assemblies. Ultrashort (\SI{<100}{\femto\second}) laser pulses are produced by a mode-locked laser system. These short pulses lack the high peak power, and are amplified via chirped pulse amplification. in a regenerative laser amplifier The resulting pulses are short (\SI{<100}{\femto\second}), while also reaching peak powers of up to \SI{1}{\peta\watt}.

In the following two subsections, a brief description of the two components of the laser system is given. These sections only highlight the basic functionality, while overlooking numerous details for brevity. All bust the most avid reader is encouraged to skip ahead to sec.~\ref{sec:spectroscopy}.

\hypertarget{femtosecond-mode-locked-oscillators}{%
\subsection{Femtosecond mode-locked oscillators}\label{femtosecond-mode-locked-oscillators}}

The core working principle of mode-locked oscillators is to fix the phase relationship between allowed amplitude modes within a laser cavity. This phase relationship enforces that constructive interference happen regularly, producing a train of ultrashort pulses. These systems typically use titanium-doped sapphire as the gain medium (Ti-sapphire), because the gain bandwidth of this medium accomodates the intrinsically large bandwidth of short laser pulses (\SI{>45}{\nano\meter}, or \SI{>3}{\tera\hertz}).

Figure~\ref{fig:mode-locking} shows an example of the first 30 longitudinal modes of a cavity coming together to form a strong pulse. In reality, pulses may combine \(10^6\) modes to maximize peak power \citep{Siegman1986}.

\begin{figure}
\hypertarget{fig:mode-locking}{%
\centering
\includegraphics{build/plots/-1297048673852454407.png}
\caption{Demonstration of how phase relationship between amplitude modes in a laser cavity can lead to a strong pulse. \textbf{a)} The first six longitudinal modes of a laser cavity. \textbf{b)} Combination of the first 30 modes of the cavity creates a very strong pulse in the center of the cavity.}\label{fig:mode-locking}
}
\end{figure}

\hypertarget{regenerative-laser-amplifiers}{%
\subsection{Regenerative laser amplifiers}\label{regenerative-laser-amplifiers}}

\hypertarget{sec:spectroscopy}{%
\section{Ultrafast spectroscopic techniques and their limitations}\label{sec:spectroscopy}}

\hypertarget{sec:microscopy}{%
\section{Electron microscopy}\label{sec:microscopy}}

\hypertarget{sec:scattering}{%
\section{Ultrafast electron scattering}\label{sec:scattering}}

\hypertarget{sec:overview}{%
\section{Overview of the dissertation}\label{sec:overview}}

\hypertarget{references}{%
\section*{References}\label{references}}
\addcontentsline{toc}{section}{References}

\bibliographystyle{unsrtnatclean}
\bibliography{references}

\hypertarget{scattering}{%
\chapter{Scattering}\label{scattering}}

Lorem ipsum dolor sit amet, consectetur adipiscing elit, sed do eiusmod tempor incididunt ut labore et dolore magna aliqua. Ut enim ad minim veniam, quis nostrud exercitation ullamco laboris nisi ut aliquip ex ea commodo consequat. Duis aute irure dolor in reprehenderit in voluptate velit esse cillum dolore eu fugiat nulla pariatur. Excepteur sint occaecat cupidatat non proident, sunt in culpa qui officia deserunt mollit anim id est laborum.

\hypertarget{sec:graphite}{%
\chapter{Momentum-resolved excitation couplings in graphite}\label{sec:graphite}}

\hypertarget{sec:snse}{%
\chapter{Ultrafast thermoelectric enhancement in tin selenide}\label{sec:snse}}

\hypertarget{metastable-phase-in-charge-density-wave-material-tase_2}{%
\chapter{\texorpdfstring{Metastable phase in charge-density wave material TaSe\(_2\)}{Metastable phase in charge-density wave material TaSe\_2}}\label{metastable-phase-in-charge-density-wave-material-tase_2}}

\hypertarget{sec:conclusion}{%
\chapter{Conclusion}\label{sec:conclusion}}

Lorem ipsum dolor sit amet, consectetur adipiscing elit, sed do eiusmod tempor incididunt ut labore et dolore magna aliqua. Ut enim ad minim veniam, quis nostrud exercitation ullamco laboris nisi ut aliquip ex ea commodo consequat. Duis aute irure dolor in reprehenderit in voluptate velit esse cillum dolore eu fugiat nulla pariatur. Excepteur sint occaecat cupidatat non proident, sunt in culpa qui officia deserunt mollit anim id est laborum.

\hypertarget{references-1}{%
\chapter*{References}\label{references-1}}
\addcontentsline{toc}{chapter}{References}

\markboth{References}{References}

\hypertarget{refs}{}
\begin{cslreferences}
\end{cslreferences}

\appendix

\backmatter
\end{document}
